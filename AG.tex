%header here
\documentclass[a4paper, 12pt]{article}
\parindent0pt


\relpenalty=9999
\binoppenalty=9999

\usepackage[margin=1in]{geometry}
\usepackage{amsmath}
\usepackage{amssymb}
\usepackage{tikz-cd}
\usepackage{longtable}


\newcommand{\ca}[1]{\mathcal{#1}}
\newcommand{\caf}{\mathcal{F}}
\newcommand{\cag}{\mathcal{G}}
\newcommand{\oxmod}{$\mathcal{O}_X$-module }
\newcommand{\oxmods}{$\mathcal{O}_X$-modules}
\newcommand{\oymod}{$\mathcal{O}_Y$-module }
\newcommand{\oymods}{$\mathcal{O}_Y$-modules}
\newcommand{\ox}{\mathcal{O}_X}
\newcommand{\oy}{\mathcal{O}_Y}
\newcommand{\oxx}{\mathcal{O}_{X,x}}
\newcommand{\oyy}{\mathcal{O}_{Y,y}}

\begin{document}

\setcounter{section}{4}

for errors or questions email: nicolatr@student.ethz.ch

\section{Sheaves of Modules}
\subsection{$\mathcal{O}(X)-modules$}
Fix a scheme X
\\






\begin{tabular}{p{4cm} p{11cm}}


sheaf of \oxmods: & A (pre)sheaf of  $\mathcal{O}(X)$-modules is a (pre)sheaf of abelian groups $\mathcal{F}$ on X together with a morphism of presheaves of sets $\mathcal{O}_X\times\mathcal{F}\longrightarrow\mathcal{F}$, which turns $\mathcal{F}(U)$ into an $\mathcal{O}_X(U)$-module for every open U$\subset$ X
\\

\end{tabular}
\\

A homomorphism of such is a morphism of presheaves $\mathcal{F}\rightarrow\mathcal{G}$ such that $\mathcal{F}(U)\rightarrow\mathcal{G}(U)$ is $\ca{O}_X(U)$-linear for each U
\\

analogously define composition, identity morphism etc. $\leadsto$ this forms a category.
\\

Take \oxmod  $\ca{F}_i$ for $i \in I$
\\
\begin{longtable}{p{.30\textwidth}  p{.70\textwidth} } 
%\begin{tabular}{p{4cm} p{12cm}}


direct product: & The direct product $\prod \ca{F}_i$ is defined by $U\longmapsto \prod(\ca{F}_i(U))$ with component-wise structure
\\

&\\

direct sum: & the direct sum $\bigoplus \ca{F}_i$ is the sheafification the presheaf $U\longmapsto \bigoplus \ca{F}_i(U)$. This is a subsheaf of $\prod\ca{F}_i$ with equality if $|I|$ is finite.
\\

&\\

tensor product: &The tensor product $\ca{F}\otimes\ca{G}$ of \oxmods $\ca{F}$ \& $\cag$ is the sheafification of the presheaf $U\longmapsto \caf(U)\otimes_{O(U)}\cag(U)$
\\

&\\

inner Hom: & The inner Hom is the sheaf of homomorphisms of \oxmods $\caf$ and $\cag$

$U\longmapsto Hom_{O_X}(\caf|_U , \cag|_U) =: \ca{H}om_{\ca{O}(X)}(\caf , \cag)(U)$
\\

%\end{tabular}
\end{longtable}


\subsection{Locally free \oxmods}

\begin{tabbing}
  Abbreviate:     \= $\ox^{(I)} := \bigoplus_{i \in I} \ox $\\
  \>$  \ox^I := \prod_{i \in I} \ox $\\
  \> $ \ox^r := \bigoplus^{r}_{i=1}\ox$
\end{tabbing}
\\

\begin{tabular}{p{4cm} p{12cm}}

free \oxmod: & An \oxmod   isomorphic to $\ox^{(I)}$ for some I is called free\\

&\\

locally free \oxmod: 

   &(a)  An \oxmod is locally free if $\forall x  \exists  x\in U \subset X open $ $\exists I : \caf|_U \cong \ox^{(I)}|_U$\\
   &(b)  It is called locally free of rank r if $\exists I : |I| = r $   and $ \forall x \exists x \in U \subset X : \caf|_U \cong \ox^{(I)}|_U$\\
\\

dual sheaf: & For $\caf$ locally free of finite rank, $\caf^{\vee} := \ca{H}om_{\ox}(\caf, \ox)$ is called the dual sheaf, also locally free of rank r.\\


invertible sheaf: & A locally free sheaf of rank 1 is called an invertible sheaf.\\


Picard group: & The set of isomorphism classes of invertible sheaves on X form an abelian group, $Pic(X)$, called the Picard group of X.\\


\end{tabular}

\subsection{\oxmod s on affine schemes}

\subsection{Quasicoherent \oxmod s }

Let X be an arbitrary scheme

Notation for global sections:
$\caf(X) = \Gamma(X, \caf) = H^0(X,\caf)$
\\

Any system of global sections $s_i \in \caf(X)$ determines a homomorohism $\ox^{(I)}\longrightarrow\caf$, $(f_i)_{i\in I }\longmapsto \sum f_i \cdot  res^{X}_{U}$\\

$\caf$ is called generated by global sections if $\exists I $: $\exists$ a surjective homomorphism:  $\ox^{(I)}\twoheadrightarrow \caf$ 
\\

\begin{tabular}{p{4cm} p{12cm}}


quasicoherent \oxmod: & An \oxmod $\caf$ is called quasicoherent if $\forall x\in X \exists$ open neighbourhood $x\in U \subset X\,  \exists I\, \exists J\, \exists $ exact sequence
  \[
\ox^{(I)}|_U \longrightarrow \ox^{(J)}|_U \longrightarrow \caf|_U \longrightarrow 0
\]
\\
\end{tabular}


locally free $\Longrightarrow$ quasicoherent





\subsection{Coherent sheaves}

\begin{tabular}{p{4cm} p{12cm}}



finitely generated \oxmod : & An \oxmod $\caf$ is called: \\
  & (a) finitely generated (or of finite type) if $\forall\,\exists x\in U \subset X $
  open:
  $\exists n < \infty$ $\exists \, \ox^n|_U \twoheadrightarrow \caf|_U$  \oxmod  homomorphism.
\\

&\\

coherent \oxmod &(b) coherent if it is finitely generated and $\forall U \subset X $ open $\forall n < \infty \, \forall$
homomorphisms $\varphi : \ox^n|_U \twoheadrightarrow \caf|_U \, ker(\varphi)$ is
finitely generated. \\
\end{tabular}

\\
%


  
  



\subsection{Functoriality}

Consider a morphism $f:X\longrightarrow Y$ with sheaves $\caf$ on X and $\cag$ on Y.

Let $\caf$ be an \oxmod and $\cag$ be an $\ca{O}_Y$-module.

$f$ comes with a homomorphism of sheaves of rings:

$f^{\flat}:\oy\longrightarrow f_{\ast}\ox  \Longleftrightarrow f^{\sharp}:f^{\ast}\oy\longrightarrow \ox$\\




\begin{tabular}{p{4cm} p{12cm}}


push-forward of a sheaf: & Make $f_{\ast}\caf$ into an \oymod by\\

  %make this
 & \begin{tikzcd}
\oy(V)\times(f_*\caf)(V) \arrow[r]\arrow[d, shift right=5ex]\arrow[d,"=", shift left=7ex] & (f_*\caf)(V) \arrow[d]  \\
\ox(f^{-1} (V))\times \caf(f^{-1})  \arrow[r, "mult."] & \caf(f^{-1}(V))
\end{tikzcd}

\\

inverse image: & the inverse image of an \oxmod is $f^{\ast}\cag:= f^{-1}\cag\otimes_{f^{-1}\oy}\ox$.  i.e.the sheafification of the presheaf \[U\longmapsto \varinjlim_{f(U)\subset V\subset Y}(\ox(U)\otimes_{\oy(V)}\cag(V))\]\\

scheme-theoretic support of $\caf$: & Let $\caf$ be a quasicoherent, finitely generated \oxmod. the scheme-theoretic support is the smallest closed subscheme $i:Y\hookrightarrow X $ such that $\caf \cong i_{\ast}i^{\ast}\caf$. Moreover $Y = \{x \in X | \caf_x \neq 0 \}$\\


 \end{tabular}
\\

\subsection{\oxmods on a projective scheme}

%%more explanation here

For any graded R-module M and any $n\in \mathbb{Z} $ we set $M(n) := M$ as R-module with grading $M(n)_d = M_{n+d}$
\\

$\widetilde{R(n)} =: \ox(n)$

clear: $\ox(0) \cong \ox$\\

\begin{tabular}{p{4cm} p{11cm}}

twisting sheaf: & $\ox(1)$ is called the twisting sheaf on $X = ProjR$

  for any \oxmod we set $\caf(n) := \caf\otimes\ox(n)$
  \\
  \end{tabular}

\subsection{Morphisms to projective spaces}

\begin{tabular}{p{4cm} p{11cm}}

very ample sheaf: &  An invertible sheaf $\ca{L}$ is called very ample (over SpecR) if $\ca{L} \cong f^{\ast}\ox(1)$ for some locally closed embedding over R $f:X\hookrightarrow \mathbb{P}^n_R $ for some n.
\\

\\

  ample sheaf:& An invertible sheaf $\ca{L}$ on a quasicompact scheme X is called ample if for any finitely generated quasicoherent sheaf $\caf$ on X there exists $n_0$ such that $n \geq n_0 $ the sheaf $\caf\otimes\ca{L}^{\otimes n}$ is generated by global sections.

  \\


  relatively ample /very ample sheaf & $\ca{L}$ an invertible sheaf with respect to $f:X\longrightarrow Y$ iff there exists an affine open covering $Y = \bigcup_{i\in I} V_i$ such that $\forall i \ca{L}|_{f^{-1}(V_i)}$ is ample/ very ample over $V_i$
  \\
\end{tabular}

\subsection{Divisors}


Assume X integral with function field K.

\begin{longtable}{p{.30\textwidth}  p{.70\textwidth} } 
%\begin{tabular}{p{4cm} p{11cm}}
	
field of rational functions on X & the constant sheaf $\ca(K)_X:= \underline{K}$ on X is called the sheaf of rational functions on X\\

&\\

	

  group of Cartier divisors & $Div(X) := \Gamma(X, \ca{K}_X^{\times}/\ox^{\times}) $\\
  & $ Div(X) = \{ ([f_x]) \in \prod_{x \in X} K^{\times}/\oxx^{\times} \; | \; \forall x \exists U \exists f \in K^{\times} \forall y \in U : f_y\cdot \ca{O}_{X,y}^{\times} = f\ca{O}_{X,y}^{\times} \}$ \\
  & $=  \{$ collections of $ f_i \in K^{\times}$ for all $i \in I$ and an open covering $ X=\bigcup U_i$ such that $\forall i,j : f_i / f_j \in \Gamma(U_i\cap U_j, \ox^{\times})   \} /$ modulo some equivalence relation\\
  & convention: The group law on Div(X) is written additively.\\
  \\

  \\
%\end{tabular}

%\begin{tabular}{p{4cm} p{11cm}}

  $\ox(D)(U)$ & For any Cartier divisor $D = ([f_x]) x \in X $ and any open $U \subset X$ set:\\
  & \[
  \ox(D)(U) = \begin{cases}
    \{0\} if U= \emptyset\\
    \bigcap_{x\in U}f_x^{-1}\oxx  else\\
    \end{cases}
    \]
    \\
    \\

    locally factorial scheme & A noetherian integral scheme such that $\forall x \in X  \oxx $ is factorial is called locally factorial.\\

    \\
%\end{tabular}

%\begin{tabular}{p{4cm} p{11cm}}
  effective cartier divisor & A cartier divisor $D = ([f_x])_x$ is called effective if $\forall x \in X: \; f_x \in \oxx $\\

  & equiv: $\forall i :  f_i \in \ox(u_i)$\\
  &$\iff \ox(-D) \subset \ox \iff \ox \subset \ox(D)$\\
  &$\iff \ox(-D)$ is a quasicoherent sheaf of ideals of $ \ox$ \\
  & $\iff $ D corresponds to a closed subscheme locally given by  one equation $f_i$ i.e. locally principal. this subscheme determines D. \\

  \\

  principal cartier divisor & A cartier divisor of the form $div(f) := (f) := ([f])_{x \in X}$ for some $f \in K^{\times}$ is called principal.\\

  \\


%\end{tabular}

\\
\newline
\\



%\begin{tabular}{p{4cm} p{11cm}}

  cartier divisor class group of X & The factor group $DivCl(X) := Div(X)/{principal divisors}$\\

  &\\
%\end{tabular}

\\


%\begin{tabular}{p{4cm} p{11cm}}


    prime cycle & an integral closed subscheme is called a prime cycle. (equiv: irred closed subset)\\

    &\\

    codimension of a prime cycle & A prime cycle's codimension is $dim \ca{O}_{X,y}$ for the generic point $y \in Y$.\\

    &\\

    cycle & A finite formal $\mathbb{Z}$-linear combination of prime cycles $\sum_y n_y y$ is called a cycle.\\

    &\\

    codimension of a cycle & If all these Y have codim d the cycle has codim d\\

    &\\

    Weil divisor & A cycle of dimension 1 is called a Weil divisor.\\

    &\\
    
  
%\end{tabular}



%\begin{tabular}{p{4cm} p{11cm}}
	principal weil divisor & A Weil divisor of the form $cyc(div(f)) = \sum ord_y(f)\cdot\overline{\{y\}}$ is called principal.\\
	
	&\\
	
	Weil divisor class group& The factor group $Z¹(X)/\{principal\} = Cl(X)$ is called the Weil divisor group. \\

	&\\

	effective weil divisor& A weil divisor $\sum n_y Y$ is effective if all $n_y \geq 0$\\
	& equiv: associated cartier divisor is effective\\
	& equiv: $\ox(-D)$ is an ideal sheaf of $\ox$\\
	
	&\\
	
	ample/very ample divisor & A divisor is called ample/very ample iff $\ox(D)$ is dito.\\
	
	&\\
	
		
  %\end{tabular}
\end{longtable}


\subsection{Differentials}

Let X=Spec B , Y=Spec A and $f:Y\longrightarrow X$ \\

\begin{tabular}{p{4cm} p{11cm}}


	A-derivation of B to M & An A-derivation of B to a B-module M is a map $d:B\longrightarrow M$ with $\forall b,b'\in B$ $\forall a  \in A$\\
	&(a)$d(b+b')=d(b)+d(b')$\\
	&(b)$d(b\cdot b')= b\cdot b' + d(b)\cdot b'$\\
	&(c)$d(a\cdot 1_B)= 0$\\

  \end{tabular}

\begin{tabular}{p{4cm} p{11cm}}


  \end{tabular}

\begin{tabular}{p{4cm} p{11cm}}


  \end{tabular}

\begin{tabular}{p{4cm} p{11cm}}


  \end{tabular}

\begin{tabular}{p{4cm} p{11cm}}


  \end{tabular}


\section{cohomology}







\end{document}



%to do:
% go from tabular to longtable

